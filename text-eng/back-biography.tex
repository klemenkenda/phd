%--------------------------------------------------------------------------------------------------
% 
\chapter{Biography}
%--------------------------------------------------------------------------------------------------

\begin{wrapfigure}{o}{0.25\textwidth}
    \vspace{-0.5cm}
    \includegraphics[width=1\linewidth]{figures/klemen_kenda_qlector.jpg} 
    \label{fig:portrait}
\end{wrapfigure}

Klemen Kenda is a researcher at Jozef Stefan Institute and Qlector d.o.o.
He obtained his diploma in physics at University of Ljubljana and is pursuing his PhD in Information and Communication Technologies at Jožef Stefan International Postgraduate School. 

His broad early working experience include development of control systems for large physics experiments at Sinchrotrone Trieste and Forschungzentrum Karlsruhe (collaboration with F2 department at JSI, later Cosylab), data analysis for online advertising at Httpool d.o.o., implementing geospacial solutions at DFG Consulting, analysing environmental software at Environmental agency of Slovenia, independent web development, teaching of physics, mathematics and programming at Primary school Cerkno and Gimnazija Jurija Vege Idrija, managing public relations at Scout association of Slovenia and several local, national and international management positions within Slovenian orienteering federation.

As a researcher, he has been involved with machine learning and stream mining of heterogeneous data sources. 
Applications of his work have been made in the fields of environmental intelligence, production planning and energy management. 
From 2011, he has contributed to several EU FP7, H2020 and Horizon Europe projects (Planetdata, Envision, NRG4CAST, Sunseed, PerceptiveSentinel, EnviroLENS, Factlog, Water4Cities, NAIADES, STAR, AquaSPICE, APRIORI, Plooto and HumAIne) and acted as a leader of several technical tasks and work packages. 

As a researcher, he has been involved in the Qlector company, firstly as QlectorLEAP developer and later as EU project leader.
Beside being used in several EU projects, his work described in the thesis has been implemented in several use cases also in industry. 
%Iskratel d.o.o.